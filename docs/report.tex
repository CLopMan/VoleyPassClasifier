%----------
%   IMPORTANTE
%----------

% Si nunca has utilizado LaTeX es conveniente que aprendas una serie de conceptos básicos antes de utilizar esta plantilla. Te aconsejamos que leas previamente algún tutorial (puedes encontar muchos en Internet).

% Esta plantilla está basada en las recomendaciones de la guía "Trabajo fin de Grado: Escribir el TFG", que encontrarás en http://uc3m.libguides.com/TFG/escribir
% contiene recomendaciones de la Biblioteca basadas principalmente en estilos APA e IEEE, pero debes seguir siempre las orientaciones de tu Tutor de TFG y la normativa de TFG para tu titulación.

% Encontrarás un ejemplo de TFG realizado con esta misma plantilla en la carpeta "_ejemplo_TFG_2019". Consúltalo porque contiene ejemplos útiles para incorporar tablas, figuras, listados de código, bibliografía, etc.


%----------
%    CONFIGURACIÓN DEL DOCUMENTO
%----------

% Definimos las características del documento y añadimos una serie de paquetes (\usepackage{package}) que agregan funcionalidades a LaTeX.

\documentclass[12pt]{report} %fuente a 12pt

% MÁRGENES: 2,5 cm sup. e inf.; 3 cm izdo. y dcho.
\usepackage[
a4paper,
vmargin=2.5cm,
hmargin=3cm
]{geometry}

% INTERLINEADO: Estrecho (6 ptos./interlineado 1,15) o Moderado (6 ptos./interlineado 1,5)
\renewcommand{\baselinestretch}{1.15}
\parskip=6pt

\providecommand{\tightlist}{%
  \setlength{\itemsep}{0pt}\setlength{\parskip}{0pt}}

% DEFINICIÓN DE COLORES para portada y listados de código
\usepackage[table,dvipsnames]{xcolor}
\definecolor{azulUC3M}{RGB}{0,0,102}
\definecolor{gray97}{gray}{.97}
\definecolor{gray75}{gray}{.75}
\definecolor{gray45}{gray}{.45}

\usepackage{tikz}
\usepackage{pgfplots}
% Soporte para GENERAR PDF/A --es importante de cara a su inclusión en e-Archivo porque es el formato óptimo de preservación y a la generación de metadatos, tal y como se describe en http://uc3m.libguides.com/ld.php?content_id=31389625. En la carpeta incluímos el archivo plantilla_tfg_2017.xmpdata en el que puedes incluir los metadatos que se incorporarán al archivo PDF cuando lo compiles. Ese archivo debe llamarse igual que tu archivo .tex. Puedes ver un ejemplo en esta misma carpeta.
\usepackage[a-1b]{pdfx}

% ENLACES
\usepackage{hyperref}
\hypersetup{colorlinks=true,
    linkcolor=black, % enlaces a partes del documento (p.e. índice) en color negro
    citecolor=black,
    urlcolor=blue} % enlaces a recursos fuera del documento en azul

% EXPRESIONES MATEMATICAS
\usepackage{amsmath,amssymb,amsfonts,amsthm}
\usepackage{booktabs}

\usepackage{txfonts}
\usepackage[T1]{fontenc}
\usepackage[utf8]{inputenc}

\usepackage[spanish, es-tabla]{babel}
\usepackage[babel, spanish=spanish]{csquotes}
\AtBeginEnvironment{quote}{\small}

% diseño de PIE DE PÁGINA
\usepackage{fancyhdr}
\pagestyle{fancy}
\fancyhf{}
\renewcommand{\headrulewidth}{0pt}
\rfoot{\thepage}
\fancypagestyle{plain}{\pagestyle{fancy}}

\usepackage{eurosym}

% DISEÑO DE LOS TÍTULOS de las partes del trabajo (capítulos y epígrafes o subcapítulos)
\usepackage{titlesec}
\usepackage{titletoc}
\titleformat{\chapter}[block]
{\large\bfseries\filcenter}
{\thechapter.}
{5pt}
{\MakeUppercase}
{}
\titlespacing{\chapter}{0pt}{0pt}{*3}
\titlecontents{chapter}
[0pt]
{}
{\contentsmargin{0pt}\thecontentslabel.\enspace\uppercase}
{\contentsmargin{0pt}\uppercase}
{\titlerule*[.7pc]{.}\contentspage}

\titleformat{\section}
{\bfseries}
{\thesection.}
{5pt}
{}
\titlecontents{section}
[5pt]
{}
{\contentsmargin{0pt}\thecontentslabel.\enspace}
{\contentsmargin{0pt}}
{\titlerule*[.7pc]{.}\contentspage}

\titleformat{\subsection}
{\normalsize\bfseries}
{\thesubsection.}
{5pt}
{}
\titlecontents{subsection}
[10pt]
{}
{\contentsmargin{0pt}
    \thecontentslabel.\enspace}
{\contentsmargin{0pt}}
{\titlerule*[.7pc]{.}\contentspage}


% DISEÑO DE TABLAS. Puedes elegir entre el estilo para ingeniería o para ciencias sociales y humanidades. Por defecto, está activado el estilo de ingeniería. Si deseas utilizar el otro, comenta las lineas del diseño de ingeniería y descomenta las del diseño de ciencias sociales y humanidades
\usepackage{multirow} % permite combinar celdas
\usepackage{caption} % para personalizar el título de tablas y figuras
\usepackage{floatrow} % utilizamos este paquete y sus macros \ttabbox y \ffigbox para alinear los nombres de tablas y figuras de acuerdo con el estilo definido. Para su uso ver archivo de ejemplo
\usepackage{array} % con este paquete podemos definir en la siguiente linea un nuevo tipo de columna para tablas: ancho personalizado y contenido centrado
\newcolumntype{P}[1]{>{\centering\arraybackslash}p{#1}}
\DeclareCaptionFormat{upper}{#1#2\uppercase{#3}\par}

% Diseño de tabla para ingeniería
\captionsetup[table]{
    format=upper,
    name=TABLA,
    justification=centering,
    labelsep=period,
    width=.75\linewidth,
    labelfont=small,
    font=small,
}

%Diseño de tabla para ciencias sociales y humanidades
%\captionsetup[table]{
%    justification=raggedright,
%    labelsep=period,
%    labelfont=small,
%    singlelinecheck=false,
%    font={small,bf}
%}


% DISEÑO DE FIGURAS. Puedes elegir entre el estilo para ingeniería o para ciencias sociales y humanidades. Por defecto, está activado el estilo de ingeniería. Si deseas utilizar el otro, comenta las lineas del diseño de ingeniería y descomenta las del diseño de ciencias sociales y humanidades
\usepackage{graphicx}
\graphicspath{{img/}} %ruta a la carpeta de imágenes

% Diseño de figuras para ingeniería
\captionsetup[figure]{
    format=hang,
    name=Fig.,
    singlelinecheck=off,
    labelsep=period,
    labelfont=small,
    font=small
}

% Diseño de figuras para ciencias sociales y humanidades
%\captionsetup[figure]{
%    format=hang,
%    name=Figura,
%    singlelinecheck=off,
%    labelsep=period,
%    labelfont=small,
%    font=small
%}


% NOTAS A PIE DE PÁGINA
\usepackage{chngcntr} %para numeración contínua de las notas al pie
\counterwithout{footnote}{chapter}

% LISTADOS DE CÓDIGO
% soporte y estilo para listados de código. Más información en https://es.wikibooks.org/wiki/Manual_de_LaTeX/Listados_de_código/Listados_con_listings
\usepackage{listings}

% definimos un estilo de listings
\lstdefinestyle{estilo}{ frame=Ltb,
    framerule=0pt,
    aboveskip=0.5cm,
    framextopmargin=3pt,
    framexbottommargin=3pt,
    framexleftmargin=0.4cm,
    framesep=0pt,
    rulesep=.4pt,
    backgroundcolor=\color{gray97},
    rulesepcolor=\color{black},
    %
    basicstyle=\ttfamily\footnotesize,
    keywordstyle=\bfseries,
    stringstyle=\ttfamily,
    showstringspaces = false,
    commentstyle=\color{gray45},
    %
    numbers=left,
    numbersep=15pt,
    numberstyle=\tiny,
    numberfirstline = false,
    breaklines=true,
    xleftmargin=\parindent,
    inputencoding = utf8,  % Input encoding
    extendedchars = true,  % Extended ASCII
    literate      =        % Support additional characters
      {á}{{\'a}}1  {é}{{\'e}}1  {í}{{\'i}}1 {ó}{{\'o}}1  {ú}{{\'u}}1
      {Á}{{\'A}}1  {É}{{\'E}}1  {Í}{{\'I}}1 {Ó}{{\'O}}1  {Ú}{{\'U}}1
      {ñ}{{\~n}}1
}

\captionsetup[lstlisting]{font=small, labelsep=period}
% fijamos el estilo a utilizar
\lstset{style=estilo}
\renewcommand{\lstlistingname}{\uppercase{Código}}


%BIBLIOGRAFÍA - PUEDES ELEGIR ENTRE ESTILO IEEE O APA. POR DEFECTO ESTÁ CONFIGURADO IEEE. SI DESEAS USAR APA, COMENTA LAS LÍNEA DE IEEE Y DESCOMENTA LAS DE APA. Si haces cambios en la configuración de la bibliografía y no obtienes los resultados esperados, es recomendable limpiar los archivos auxiliares y volver a compilar en este orden: COMPILAR-BIBLIOGRAFIA-COMPILAR

% Tienes más información sobre cómo generar bibliografía y CONFIGURAR TU EDITOR DE TEXTO para compilar con biber en http://tex.stackexchange.com/questions/154751/biblatex-with-biber-configuring-my-editor-to-avoid-undefined-citations , https://www.overleaf.com/learn/latex/Bibliography_management_in_LaTeX y en http://www.ctan.org/tex-archive/macros/latex/exptl/biblatex-contrib
% También te recomendamos consultar la guía temática de la Biblioteca sobre citas bibliográficas: http://uc3m.libguides.com/guias_tematicas/citas_bibliograficas/inicio

% CONFIGURACIÓN PARA LA BIBLIOGRAFÍA IEEE
\usepackage[backend=biber, style=ieee, isbn=false,sortcites, maxbibnames=5, minbibnames=1]{biblatex} % Configuración para el estilo de citas de IEEE, recomendado para el área de ingeniería. "maxbibnames" indica que a partir de 5 autores trunque la lista en el primero (minbibnames) y añada "et al." tal y como se utiliza en el estilo IEEE.

%CONFIGURACIÓN PARA LA BIBLIOGRAFÍA APA
%\usepackage[style=apa, backend=biber, natbib=true, hyperref=true, uniquelist=false, sortcites]{biblatex}
%\DeclareLanguageMapping{spanish}{spanish-apa}

% Añadimos las siguientes indicaciones para mejorar la adaptación de los estilos en español
\DefineBibliographyStrings{spanish}{%
    andothers = {et\addabbrvspace al\adddot}
}
\DefineBibliographyStrings{spanish}{
    url = {\adddot\space[En linea]\adddot\space Disponible en:}
}
\DefineBibliographyStrings{spanish}{
    urlseen = {Acceso:}
}
\DefineBibliographyStrings{spanish}{
    pages = {pp\adddot},
    page = {p.\adddot}
}

\addbibresource{references.bib} % llama al archivo bibliografia.bib en el que debería estar la bibliografía utilizada


%-------------
%    DOCUMENTO
%-------------

\begin{document}
\pagenumbering{roman} % Se utilizan cifras romanas en la numeración de las páginas previas al cuerpo del trabajo

%----------
%    PORTADA
%----------
\title{Práctica Final}
\author{Álvaro Guerrero Espinosa (100472294)\\
        César López Mantecón (100472092)\\
        Paula Subías Serrano (100472119)\\
        Irene Subías Serrano (100472108)\\}

\makeatletter
\begin{titlepage}
    \begin{sffamily}
    \color{azulUC3M}
    \begin{center}
        \begin{figure}[H] %incluimos el logotipo de la Universidad
            \makebox[\textwidth][c]{\includegraphics[width=16cm]{Portada_Logo.png}}
        \end{figure}
        \vspace{2.5cm}
        \begin{Large}
            Grado en Ingeniería Informática\\
            \@date\\
            \vspace{2cm}
            \textsl{Inteligencia Artificial en las Organizaciones}\\
            \bigskip
        \end{Large}
        {\Huge ``\@title''}\\
        \vspace*{0.5cm}
        \rule{10.5cm}{0.1mm}\\
        \vspace*{0.9cm}
        {\LARGE\@author}
        \vspace*{1cm}
    \end{center}
    \vfill
    \color{black}
    % si nuestro trabajo se va a publicar con una licencia Creative Commons, incluir estas lineas. Es la opción recomendada.
    \includegraphics[width=4.2cm]{creativecommons.png}\\ %incluimos el logotipo de creativecommons
    Esta obra se encuentra sujeta a la licencia Creative Commons \textbf{Reconocimiento - No Comercial - Sin Obra Derivada}
    \end{sffamily}
\end{titlepage}
\makeatother

\newpage %página en blanco o de cortesía
\thispagestyle{empty}
\mbox{}

%----------
%    ÍNDICES
%----------

%--
% Índice general
%-
\tableofcontents
\thispagestyle{fancy}

\newpage % página en blanco o de cortesía
\thispagestyle{empty}
\mbox{}

%--
% Índice de figuras. Si no se incluyen, comenta las lineas siguientes
%-
 \listoffigures
 \thispagestyle{fancy}

 \newpage % página en blanco o de cortesía
 \thispagestyle{empty}
 \mbox{}

%--
% Índice de tablas. Si no se incluyen, comenta las lineas siguientes
%-
\listoftables
 \thispagestyle{fancy}

 \newpage % página en blanco o de cortesía
 \thispagestyle{empty}
 \mbox{}


%----------
%    TRABAJO
%----------
\clearpage
\pagenumbering{arabic} % numeración con múmeros arábigos para el resto de la publicación

    \chapter{Introducción}
    \label{chap:intro}

    En este documento se recoge el desarrollo y conclusiones del proyecto final
    de la asignatura \textit{Inteligencia artificial en las organizaciones}.

    En el ámbito del deporte, el análisis del rendimiento ha sido
    tradicionalmente una tarea realizada por entrenadores y analistas, quienes
    observan los partidos para evaluar las acciones y estrategias de los
    jugadores. Sin embargo, este proceso es subjetivo, requiere tiempo y puede
    verse afectado por sesgos humanos. Con los avances recientes en
    inteligencia artificial (IA) y visión por computadora, se abren nuevas
    oportunidades para automatizar y mejorar el análisis deportivo,
    proporcionando una evaluación más objetiva, detallada y rápida.

    El voleibol es un deporte caracterizado por su velocidad, coordinación de
    varios jugadores y variedad táctica. La evaluación precisa de las acciones
    es una tarea fundamental para optimizar el desempeño tanto individual como
    colectivo. En este contexto, una IA capaz de reconocer y analizar
    secuencias del juego tiene el potencial de transformar la forma en la que
    se estudian los partidos, optimizando el proceso. 

    Este trabajo de investigación explora el desarrollo y aplicación de una
    inteligencia artificial entrenada específicamente para evaluar acciones de
    voleibol en vídeo. La propuesta combina algoritmos ámpliamente
    desarrollados para el reconocimiento de imágenes como YOLO~\cite{YOLO} y
    técnicas de análisis deportivo.

 

    \chapter{Estado del arte}
    \label{chap:estadoarte}
    En el voleibol español se emplean principalmente dos
    métodos para la toma de estadísticas de partidos según cuándo se realice:
    \begin{itemize}
        \item Toma de estadísticas en tiempo real: durante los partidos una o
        varias personas toman estadísticas manualmente o a través de
        herramientas especializadas.
        \item Toma de estadísticas en diferido: se graba el partido y
        posteriormente se revisa la grabación tomando estadísticas.
    \end{itemize}

    Ambas aproximaciones presentan claros problemas, especialmente hablando del
    groso de equipos federados que no cuentan con recursos suficientes para que
    esta actividad presente ventajas frente al esfuerzo que requiere. Hablando
    en concreto de Madrid, existen 99 clubes~\cite{directorio-clubes} y más de
    6000 deportistas~\cite{estadisticasFEVB}; mientras que sólamente hay
    201 técnicos (entrenadores y auxiliares)~\cite{estadisticasFEVB}. Esto
    refleja el claro desbalance que existe entre técnicos y deportistas, lo que
    hace imposible que muchos clubes cuenten con los recursos suficientes para
    la toma de estadísticas de forma eficiente.

    \section{Toma de estadísticas en tiempo real}
    
    Esta aproximación se suele tomar en equipos profesionales y a través de
    herramientas específicas. La toma de estadísticas suele realizarse por 3 o
    5 técnicos para su posterior publicación. Esto se hace mediante el programa
    \textit{Data volley}~\cite{datavoley}, un software especializado cuya
    licencia más económica parte de los 299\euro. 

    En equipos no profesionales, dependiendo del nivel de la competición, se
    suele tener un primer y un segundo entrenador. El primer entrenador dirige
    al equipo durante el encuentro y el segundo toma unas estadísticas
    reducidas y asesora al primero en otros menesteres. En equipos donde no
    existe la figura del segundo entrenador lo normal es no tomar estadísticas
    durante el encuentro.

    \section{Toma de estadísticas en diferido}

    La toma de estadísticas en diferido es la aproximación habitual para
    equipos con un único entrenador. Esta forma de tomar estadísticas presenta
    2 problemas: la falta de tiempo y la iniciativa individual.

    Respecto a la falta de tiempo, es importante entender que el voleibol en
    españa es un deporte minoritario. Además, un partido de voleibol puede durar
    entre 45 y 180 minutos según la categoría. El análisis del metraje y la
    toma de estadísticas puede multiplicar este tiempo por un factor de entre 3
    y 5 de manera habitual. Esto sumado a que la mayoría de entrenadores no
    tienen el voleibol como fuente de ingresos principal y que deben combinar
    su actividad como técnicos deportivos con otras actividades se
    traduce en que a muchos entrenadores les resulte imposible invertir su
    tiempo en revisar las grabaciones y tomar estadísticas de sus equipos.

    Adicionalmente, la mayoría de pabellones y clubes no cuentan con la
    infraestructura (cámaras, trípodes y espacio) necesaria para obtener
    grabaciones de buena calidad que puedan ser fácilmente analizadas. Por esto, 
    un gran porcentaje de los entrenadores que realizan la toma de estadísticas
    en diferido lo hacen por iniciativa propia. Esto desincentiva la actividad,
    haciendo de la toma de estadísticas en diferido un lujo que pueden
    permitirse sólo los clubes con mayor poder adquisitivo para facilitar la
    infraestructura a sus entrenadores.
    
    
    \chapter{Objetivo}
    \label{chap:metodos}

    El objetivo principal de este proyecto es desarrollar un sistema que
    permita la toma de estadísticas de voleibol a partir de secuencias de
    vídeo, ofreciendo una solución accesible y eficiente para clubes con
    recursos limitados. Este sistema busca automatizar el análisis de partidos,
    reduciendo significativamente la carga de trabajo manual que actualmente
    recae sobre los entrenadores, especialmente en equipos no profesionales
    donde el tiempo y los recursos son escasos.

    Dado que el estudio se trata de una primera aproximación a este problema,
    el modelo se centra en el análisis y evaluación del gesto técnico de
    colocación o armado. Se ha elegido esta acción por tratarse de la base de
    todos los sistemas de juego de este deporte, además de ser la más
    determinante para lograr la efectividad en el juego.

    Además, se pretende que esta herramienta sea adaptable a diferentes niveles de
    competición, considerando las limitaciones en la calidad de las grabaciones y
    la infraestructura de los clubes. Al facilitar la recopilación de estadísticas
    precisas y útiles, este proyecto puede contribuir al crecimiento y la
    profesionalización del voleibol en España, fomentando la igualdad de
    oportunidades entre equipos con diferentes capacidades económicas.


    \chapter{Análisis de datos}
    \label{cahp:datos}
    Para la realización del proyecto se han utilizado dos bases de datos que
    etiquetan vídeos de voleibol profesional. La primera~\cite{dataset1} contiene 4830
    secuencias extraídas de 55 vídeos distintos, cada una compuesta de 41
    \textit{frames}. Para cada secuencia se etiqueta la acción realizada durante
    la secuencia, con uno de los siguientes valores: \verb!l-spike!,
    \verb!r_spike!, \verb!r-pass!, \verb!l_winpoint!, \verb!l_set!,
    \verb!l-pass!, \verb!r_winpoint! o \verb!r_set!. Además, también se etiqueta
    la posición y acción de cada jugador durante los frames 11 a 30. Esto se
    hace con una 10-tupla con los valores ID del jugador, \verb!xmin!,
    \verb!ymin!, \verb!xmax!, \verb!ymax!, ID del frame en su video,
    \verb!lost!, \verb!grouping!, \verb!generated! y tipo. \verb!xmin!,
    \verb!ymin!, \verb!xmax! e \verb!ymax! definen el rectangulo en el que se
    encuentra el jugador, y el tipo puede tomar uno de los siguientes valores:
    \verb!waiting!, \verb!setting!, \verb!blocking!, \verb!spiking!,
    \verb!standing!, \verb!falling!, \verb!jumping!, \verb!moving! o
    \verb!digging!. Por último, \verb!lost! indica si el rectangulo anotado está
    fuera del campo de visión, \verb!grouping! indica si el jugador está
    involucrado en la acción principal, y \verb!generated! indica si el
    rectangulo fue automáticamente interpolado.

    \begin{table}[H]
        \begin{tabular}{@{}cc@{}}
            \toprule
            Acción & Num. Instancias\\
            \midrule
            Colocación    & 1276\\
            Remate        & 1265\\
            Pase          & 1627\\
            Punto Ganador &  662\\
            \bottomrule
        \end{tabular}
        \caption{Distribución de secuencias por gesto de la base de datos.}
    \end{table}

    La segunda base de datos~\cite{dataset2} utiliza el mismo conjunto de videos y secuencias,
    pero etiqueta la posición $x$ e $y$ del centro de la bola en cada frame de
    cada secuencia. Debido a esto, se puede combinar con la primera base de
    datos para obtener todos los datos que necesitamos. En los casos en los que
    la bola no se puede ver claramente (ya sea por que está oculta o porque se
    mueve demasiado rápido), la etiqueta corresponderá con la localización
    estimada de la bola.

    \chapter{Metodología}
    \label{chap:metodologia}
    En este capítulo se describirán los difentes procesos llevados a cabo para la construcción y evaluación del modelo. Todos los \textit{scripts} mencionados han sido incluídos en el \hyperref[anexo]{Anexo}.

    \section{Filtrado de datos y etiquetado}

    Este proyecto se centrará en uno de los gestos técnicos, la colocación o
    armado, con el fin de reducir el alcance de la práctica. Los datos obtenidos
    de la primera base de datos han sido filtrados y reducidos a las secuencias
    de las clases \verb!r_set! o \verb!l_set!, utilizando un programa de
    \verb!python!. Esto reduce el número de
    secuencias a 1276.

    Cada secuencia ha sido etiquetada manualmente en las clases positiva
    (representada con el símbolo \verb!+!) y negativa (representada con el
    símbolo \verb!-!) siguiendo el siguiente criterio:
    \begin{itemize}
        \item \textbf{Positiva:} se clasifica una colocación como positiva si
        el jugador ha conseguido realizar un pase con la yema de los dedos de
        ambas manos.
        \item \textbf{Negativa:} se clasifica una colocación como negativa si
        el jugador no ha logrado realizar un pase con la yema de los dedos,
        golpeándo el balón con el antebrazo o empleando una sóla mano.
    \end{itemize}
    
    Dado que existen un gran número de secuencias positivas frente a las
    negativas, se ha usado un \textit{script} de \texttt{bash} (ver
    \ref{preclasificado}) para realizar un preclasificado de las secuencias.
    Además, ha sido necesario modificar la segunda base de datos para incluir
    el tamaño de la \textit{bounding box} para cada balón. Esto último se ha
    hecho a través de un programa de \texttt{Python}. Se ha escogido un 
    tamaño genérico igual para todas las instancias ($20px\times20px$).

    Las secuencias se han separado en dos carpetas, cada una conteniendo las
    imágenes de cada una de las clases. Adicionalmente, para cada carpeta se
    han dividido las secuencias en subconjuntos de entrenamiento y evaluación.
    Para esto se ha usado un programa en \texttt{Python}.

    Adicionalmente se ha creado una clase más denominada \textit{wait} para
    aquellos frames que no aporten información relativa al tipo de acción. Se
    compararán los resultados obtenidos empleando y sin emplear esta clase.

    \section{\textit{Oversampling} y \textit{Undersampling}}

    Los datos cuentan con un desbalanceo muy fuerte hacia la clase positiva: tan
    solo un $2.59\%$ de las instancias están etiquetadas como negativas. Debido
    a esto, se ha decidido realizar tanto un \textit{oversampling} de la clase
    minoritaria como un \textit{undersampling} de la clase mayoritaria. Para la
    realización del \textit{undersampling}, se ha decidido eliminar un $30\%$ de
    las instancias de la clase positiva, seleccionadas aleatoriamente.

    \section{Entrenamiento}

    \section{Evaluación}

    \chapter{Análisis de resultados}
    \label{chap:resultados}

    \chapter{Trabajos futuros}
    \label{chap:future}

    \chapter{Conclusiones}
    \label{chap:conclusion}

    %----------
    %    BIBLIOGRAFÍA
    %----------

    %\nocite{*} % Si quieres que aparezcan en la bibliografía todos los documentos que la componen (también los que no estén citados en el texto) descomenta está lína

    \clearpage

    \phantomsection
    \addcontentsline{toc}{chapter}{Bibliografía}
    \label{chap:bibliography}
    \setquotestyle[english]{british} % Cambiamos el tipo de cita porque en el estilo IEEE se usan las comillas inglesas.
    \printbibliography

    %----------
    %    ANEXOS
    %----------

    % Si tu trabajo incluye anexos, puedes descomentar las siguientes lineas
    \phantomsection
    \chapter*{Anexo}
    \label{anexo}
    \pagenumbering{gobble} % Las páginas de los anexos no se numeran

    \phantomsection
    \section*{Código Python para filtrar la base de datos de acciones}
    \label{filtrado_script}
    \lstinputlisting[language=Python]{../scripts/filter.py}

    \phantomsection
    \section*{Código Python para añadir el tamaño de \textit{bounding box}}
    \label{mod_ball}
    \lstinputlisting[language=Python]{../scripts/process_ball_anotations.py}

    \phantomsection
    \section*{Código Python para la particición de las anotaciones en ficheros}
    \lstinputlisting[language=Python]{../scripts/split_lines.py}

    \phantomsection
    \section*{Código bash para el preclasificado}
    \label{preclasificado}
    \lstinputlisting[language=bash]{../scripts/preclasificado.sh}

    \phantomsection
    \section*{Código Python para la división por carpetas}
    \label{div_carpetas}
    \lstinputlisting[language=Python]{../scripts/split_videos.py}
\end{document}
